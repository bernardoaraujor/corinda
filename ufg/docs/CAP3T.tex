%%%%%%%%%%%%%%%%%%%%%%%%%%%%%%%%%%%%%%%%%%%%%%%%%%%%%%%%%%%%%%%%%%%%%%%%
% CAP�TULO 3
\chapter{SISTEMA, MODELO E OTIMIZA��O (xx/xx/2017)}\label{chapter:cap3}

\section{Sistema}

Aqui voc� deve definir o que � sistema. Pode utilizar o livro do Medina e outros.

\section{Modelo}

Aqui voc� deve definir o que � modelo. Pode utilizar o livro do Medina e outros.

\subsection{Constru��o da modelagem do motor de corrente cont�nua}

Aqui voc� deve falar somente o necess�rio para que seu leitor entenda o que � o modelo do motor de corrente cont�nua, dado as express�es que o define.

\section{Processo de otimiza��o}

Aqui voc� deve falar sucintamente sobre o processo de otimiza��o. Pegue o material comigo.

\subsection{Otimiza��o aplicada ao controle do motor de corrente cont�nua}

Aqui voc� fala do trabalho do M�rcio, Douglas e Rafael, descrevendo o que eles fizeram como processo de otimiza��o.

\section{Considera��es}

Veja as express�es (\ref{Equ1}) e (\ref{Equ2}), que servem de exemplo de como inserir express�es matem�tica.

\begin{equation}
v_{a}(t) = R_{a}\cdot i_{a}(t) + L_{a} \frac{di_{a}(t)}{dt} + e_{g}(t)
\label{Equ1}
\end{equation}

\begin{equation}
v_{f}(t) = R_{f}\cdot i_{f}(t) + L_{f} \frac{di_{f}(t)}{dt}
\label{Equ2}
\end{equation}